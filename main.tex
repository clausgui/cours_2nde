\documentclass[a4paper,12pt]{article}
\usepackage{color}
\usepackage[utf8]{inputenc}
\usepackage[pdftex,final]{graphicx} % avant {babel} sinon pb avec MetaPost
\usepackage[frenchb]{babel}
  \frenchbsetup{StandardLists=true}
  \renewcommand{\labelitemi}{\textbullet}

\usepackage[tmargin=1cm,bmargin=1cm,hmargin=1cm]{geometry}
\usepackage{multido}

\setlength{\headheight}{0pt}
\definecolor{gris}{gray}{0.5}
\newcommand{\points}[1]
  {\raisebox{-.1cm}[7mm]{\makebox[#1]{\textcolor{gris}{\dotfill}}}}
\newcommand{\lignepoints}
  {\raisebox{-.1cm}[7mm]{}\makebox[0pt]{}\textcolor{gris}{\dotfill}\makebox[0pt]{}}
\newcommand{\coords}[2]{\begin{pmatrix}#1\\#2\end{pmatrix}}
\newcommand{\lignespoints}[1]{\multido{}{#1}{\\\lignepoints}}

\newcounter{exo}
\setcounter{exo}{1}

\newcommand{\ex}[2][]{
  \vspace{1ex}

  \noindent
  \textbf{\underline{Exercice \arabic{exo}} #1\\ }
   \stepcounter{exo}
  \noindent #2
}

\pagestyle{empty}

\begin{document}

        Classe de 2NDE1  \hfill {\large \textbf{Devoir Maison pour le 15/01/2018}} \hfill G. CLAUS
        
        \vspace{1mm}
        \hrule
        \vspace{1mm}
        Nom Prénom :  \lignepoints{} \\
        
        \ex{
       Le standard d’un cabinet médical dispose de deux lignes de téléphone. \\
    On considère les événements :    
    \begin{itemize}
        \item $O_1$ : « La 1\up{ère} ligne est occupée ».
        \item $O_2$ : « La 2\up{ème} ligne est occupée ».
    \end{itemize}
    Une étude statistique montre que :
    \begin{itemize}
        \item $P(O_1) = 0,4$
        \item $P(O_2) = 0,3$
        \item $P(O_1 \cap O_2 ) = 0,2$
    \end{itemize}
    Calculer la probabilité des événements suivants.
    \begin{enumerate}
        \item « La ligne 1 est libre ».
        \item « Au moins une des lignes est occupée ».
        \item « Au moins une des lignes est libre ».
    \end{enumerate}
    }
    
    \ex{
    Dans un pays européen, 12\,\% des moutons sont atteints par une maladie. 
    Un test de dépistage de cette maladie vient d'être mis sur le marché mais il n'est pas totalement fiable.

    Une étude a montré que quand le mouton est malade le test est positif dans 93\,\% des cas ; 
    quand le mouton est sain, le test est négatif dans 97\,\% des cas.

    On note $M$ si le mouton est malade et $S$ s'il est sain. On note $P$ si le test est positif, $N$ si le test est négatif.

    \begin{enumerate}
        \item Compléter l'arbre de probabilité suivant :\\ 
         \item Si on prend un mouton au hasard, quelle est la probabilité que le test soit positif ?
    \end{enumerate}
    }
        
\end{document}

